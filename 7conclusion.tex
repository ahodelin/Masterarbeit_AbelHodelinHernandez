\chapter{Fazit} \label{ch:conclussion}

Innerhalb dieser Masterarbeit wurden die Biosignaldaten aus der Routineversorgung in der Universitätsmedizin Mainz mit den \ac{fhir}-Profilen des Erweiterungsmoduls Intensivmedizin zusammengefügt, um diese Daten wissenschaftlich weiterzunutzen. Um dieses Ziel zu erreichen wurden bestimmte Parameter der \ac{fhir}-Profile des Modul \ac{icu} zusammen mit Parameter des \ac{pdms}-System der Universitätsmedizin Mainz in eine \ac{db} importiert. Mit Hilfe von \ac{sql}-Abfragen und \ac{regex} wurden manche interessante Biosignalparameter erkannt. 

Eine der erkannten Problematik für die Durchführung dieses Projekts war den Mangel an Standards und viele Unregelmäßigkeiten bei der Benennung der Variablen die, die Biosignaldaten im \ac{pdms}-System speichern. Andererseits besitzen viele Biosignalen im System keine Beschreibung im Backend, nämlich in der \ac{db} von \ac{copra}. Manche erkannte Biosignaldaten beinhalten keine Masseneinheiten oder diese sind nicht dieselbe wie in den \ac{fhir}-Profilen, sodass einige Vitalparameter am Ende nicht berücksichtigen könnten.

Einer der positiven Aspekte dieser Arbeit ist die Erkennung von vielen interessanten Biosignaldaten in dem \ac{pdms}-System, die mit bestimmten \ac{fhir}-Profilen verlinkt wurden und nach diesem Prozess wurde eine Dokumentation und Planung für die spätere Implementierung der Überführung der Biosignaldaten aus \ac{copra} in die \ac{fhir}-Profile. Ein interessanter Punkt zu berücksichtigen ist der Umfang des Erweiterungsmoduls Intensivmedizin. Obwohl dieses Erweiterungsmodul mit circa 81 Profile das größte der \ac{mii} ist, sind die meisten Profile strukturell ähnlich aufgebaut. Diese Eingenschaft reduziert die Komplexität und Aufwendigkeit der Arbeit mit den Profilen. 

%sind die Kollaboration der \ac{sme} der \ac{pdms}-Abteilung an der Universitätsmedizin für die Weitergabe von Information des \ac{copra}-Systems und Validierung der generierten Datensätzen für die Überführung der Daten von \ac{copra} in die \ac{fhir}-Profile.


