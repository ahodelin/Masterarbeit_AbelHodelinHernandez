\chapter{Theoretische Grundlagen} \label{ch:theobasis}

In dem folgenden Kapitel werden die Grundlagen zu den wichtigsten Konzepten, Fördermaßnahmen und Technologien näher betrachtet, die innerhalb dieser Arbeit im weiteren Verlauf angewendet werden. Zu Beginn wird das Konzept der Interoperabilität und die Erweiterung im Bereich des Gesundheitswesens präsentiert. Die Diskussion der Fördermaßnahme \glqq \ac{mii}\grqq{} mit seinem Kerndatensatz und das Erweiterungsmodul Intensivmedizin folgt im Anschluss. Die \ac{it}-Standards, wie \ac{hl7}-\ac{fhir} und die dazu benutzten Datenaustauschformate, \ac{xml} und \ac{json}, werden nachfolgend erläutert. In einer weiteren Sektion wird sich die Arbeit mit den standardisierten Codesystemen \ac{snomedct}, \ac{loinc} und \ac{ieee} beschäftigen. Das \ac{pdms} und das in Deutschland etablierte Werkzeug \ac{copra} werden präsentiert. Ein Überblick über das \ac{dw}-System wird im weiterem Verlauf der Arbeit verschafft. Anschließend werden Aspekte des Data Mappings und Pattern Matchings in zwei Sektionen erläutert.
