\section{\acl{pdms}} \label{sec:pdms}

Das \acf{pdms} unterstützt die klinische Dokumentation auf Intensivstationen \glqq\acf{icu}\grqq{} und hat damit nachweisbare Auswirkungen auf die Vollständigkeit der Patientenakten, den Zeitaufwand für die Dokumentation und die Erhöhung der Patientenqualitätssicherung \cite{pdmsfinanc, pdmsimplem, pdmsicu}. Dieses System umfasst auch Komponenten der computergestützten Auftragserfassung für Ärzte \glqq\ac{cpoe}\grqq{}, denn die Daten aus diesem System sind entscheidend für verschiedene automatisierte Workflows \cite{pdmsfinanc, pdmsicu}. Das \ac{pdms} bietet auch eine spezifische Funktionalität für die Dokumentation der \ac{drg}, welche alle relevanten Daten für die Codierung erfasst \cite{pdmsfinanc, pdmsimplem}.

In Deutschland waren im Jahr 2021 ca. 15 \ac{pdms}-Anbieter bekannt \cite{pdmsgermany}. Unter diesen Anbietern befindet sich der \acf{copra}, welcher an der Universitätsmedizin Mainz und Standort der Realisierung dieses Projekts benutzte \ac{pdms} ist.

\subsection{\acsu{copra}}

\acs{copra} System GmbH ist seit 1993 einer der führenden Anbieter von \acp{pdms} in Deutschland \cite{copradosing, copra}. Dessen Hauptprodukt ist das zertifizierte Medizinprodukt \ac{copra} in der Version 6 \glqq\ac{copra}6\grqq{}. Mit seinen vier Anwendungsgebieten: Ärzte und Ärztinnen, Pflege, Controlling und \ac{it}-Abteilung ist \ac{copra} ein \ac{pdms} für die Dokumentation von Behandlung und Pflege geeignet \cite{copra}. Aus diesem Grund wird \ac{copra} als \ac{pdms} seit 2007 an der Universitätsmedizin Mainz etabliert \cite{copraplaces}.

Für die Ärzte und Ärztinnen bietet das \ac{copra}-System eine transparente und vereinfachte Möglichkeit für die Dokumentation aller relevanten Daten einer Behandlung, denn alle Befunde der behandelnden Personen können in \ac{copra} eingesehen werden, darüber hinaus werden ärztliche Anordnungen dokumentiert und freigeschaltet \cite{copra}.

Mit den sogenannten Arbeitslisten für die Pflege werden die bereits erfolgten und nicht erfolgten Behandlungsschritte in \ac{copra} angezeigt. Außerdem werden die Kurven der Vitalparameter durch die automatische Übernahme von Werten aller an Patienten angeschlossenen Geräte aufgezeichnet. \cite{copra}.

Die Dokumentation der gesammelten Aktivitäten in \ac{copra} werden exportiert, sodass Therapieverfahren und Maßnahmen durch das Controlling ermittelt werden \cite{copra}.

\ac{copra} nutzt etablierte \ac{it}-Technologien, wie Microsoft \acs{sql} Server und .NET Framework, ist skalierbar, virtualisierbar und bietet Freiheiten bei der Gestaltung der Infrastruktur, sodass es für die \ac{it}-Abteilungen attraktiv ist \cite{copra}.

Durch die Freiheiten bei der Gestaltung der Infrastruktur von \ac{copra} ist ein multidimensionales Datenmodell wie in der \ref{fig:copraschema} möglich. 

Das Schema der \ref{fig:copraschema} ist die Darstellung des Datenmodells des Bereichs für die Biosignaldaten an der Universitätsmedizin Mainz, denn das \ac{copra}-System beinhaltet ein Schema mit mehr als 260 Tabellen insgesamt in dem Release 44.0 der Version 1 von 20.10.2015 \cite{copradoc}. Dieses Modell sollte aber nicht mit einem Sternschema in einem \ac{dw} (\ref{subsec:datamodel}) verwechselt werden, denn bei \ac{copra} stellt die Tabelle \texttt{co6\_medic\_data\_patient} mit der Basisinformation der behandelnden Personen, in der \ac{dw}-Theorie, eine Dimension dar, und die weiteren Tabellen sammeln Metadaten und die Ergebnisse der Messungen der medizinischen Geräte (Werttabellen), also die Fakten, und beinhalten auch die Hauptschlüssel von \texttt{co6\_medic\_data\_patient} als Fremdschlüssel.

\clearpage

\begin{figure}[ht]
	\centering
	\includegraphics[height=8.5cm]{figures/copra_data_model_data}
	\caption[Datenmodell von \acs{copra}]{Datenmodell des für diese Arbeit benutzten Teils vom \ac{copra}-System an der Universitätsmedizin Mainz. Die Tabelle \texttt{co6\_medic\_data\_patient} beinhaltet die Information der behandelnden Personen. Die Tabellen \texttt{co6\_data\_decimal\_6\_3}, \texttt{co6\_medic\_data\_pressure} und \texttt{co6\_data\_string} speichern die Ergebnisse der Messungen oder Techniken. Die Tabelle \texttt{co6\_data\_object} beinhaltet die Schlüssel aller Objekte, z. B. Patienten und abstrakte Elemente, wie die Arztbriefe oder Scores.
	Die Tabellen \texttt{co6\_config\_variables} und \texttt{co6\_config\_variable\_types} beinhalten die Metadaten der gespeicherten Information in \ac{copra}.}
	\label{fig:copraschema}
\end{figure}


Das Teil der \ac{copra}-\ac{db} ist zugleich eine der Datenquelle im Staging Bereich des \ac{dw} (\ref{sec:dw}) in dem \ac{diz} an der Universitätsmedizin Mainz. Noch dazu stellt dieses benutzte Teil im \ac{copra}-System ein \ac{dm} dar, also ein beschränktes \acf{dw} (\ref{sec:dw}) und somit integriert es Daten aus verschiedenen Datenquellen, in diesem Fall die unterschiedlichen  Messungen und Beobachtungen von den verschiedenen Biosignaldaten, und stellt damit diese Daten zu vielfältigen Analysezwecken zur Verfügung bereit \cite{planungdatawarehouse, dwbauer}. %Damit besitzt \ac{copra} ein \ac{olap}-System (\ref{subsec:olap}) für die Aufbereitung der Daten \cite{planungdatawarehouse, dwbauer}.