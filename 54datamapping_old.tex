\section{Data Mapping der Biosignaldaten aus \\ \acs{copra} mit den \acs{fhir}-Profilen} \label{sec:datamappingicucopra}

Nach der Analyse der \ac{fhir}-Profile des Erweiterungsmoduls \glqq Intensivmedizin\grqq{} und des \ac{copra}-Datenmodells wurden die Daten definiert und ein Pattern Matching-Prozess für die Zuordnung der Biosignaldaten aus \ac{copra} mit den \ac{fhir}-Profile. Während dieses Verfahrens entstand einen Datensatz. Dieser wurde validiert und für die Zuweisung der Quell- und Zielfelder, und die Programmierung der Transformationsregeln benutzt.

\subsection{Datendefinition} \label{subsec:datadef}

Nach der Analyse des Datenmodells (\ref{fig:copraschema}) wurden die Tabelle \texttt{mii\_icu} (\ref{sec:fhirprofs}) zusammen mit den Tabellen von \ac{copra} (\ref{sec:configvarcopra}) untersucht, um zu erkennen, welche \ac{copra}-Tabellen und Spalten mit den Elementen der \ac{fhir}-Profile zu vernetzen sind.

Die Tabelle \texttt{mii\_icu} (\ref{tab:miiicu}) der \ac{db} \texttt{mii\_copra} (\ref{sec:database}) beinhaltet die Information der \ac{fhir}-Profile des Moduls \ac{icu} des Kerndatensatzes der \ac{mii} und somit die Datenstruktur und Parameter des Zielsystems.

Das Quellsystem dieser Arbeit sind die Tabellen der \ac{copra}-Instanz des Staging Bereichs des \ac{dw} des \ac{diz} an der Universitätsmedizin Mainz. Die Tabelle \texttt{co6\_config\_variable\_types} (\ref{tab:co6confvartype}) beinhaltet die Datentypen und Namen der Werttabelle, in der die Werte der Biosignaldaten gespeichert sind. In der Tabelle \texttt{co6\_config\_variables} (\ref{tab:co6confvar}) sind die am Standort angegebenen \ac{copra}-Namen und Beschreibungen der Messungen oder angewandten Techniken zusammen mit deren Maßeinheiten. \texttt{co6\_medic\_patient} (\ref{tab:patient}) beinhaltet die Basisdaten der Patienten, wie die Patientennummer. Die Werte und Zeiten der Messungen und Beobachtungen für die Überführung der Daten in die \ac{fhir}-Profile befinden sich in den Werttabellen \texttt{co6\_data\_decimal\_6\_3}, \texttt{co6\_data\_string} (\ref{tab:valuetab}) und \texttt{co6\_medic\_pressure} (\ref{tab:valuepress}).

Für das Weiterlaufen des Projekts wurden nur die Konfigurationsvariablen ausgewählt, die einen Patientenbezug oder Fallbezug besitzen, mit mindesten 1000 validierten, aktuellen und nicht gelöschten Datensätzen im \ac{copra}-System, denn unter dieser Gruppe befinden sich die Konfigurationsvariablen, die den Biosignaldaten entsprechen.

\subsection{Zuweisung der Quell- und Zielfelder} \label{sec:patternmatchingicucopra}

Nach der Definition der Daten wurden die Quell- und Zielfelder zugewiesen. Um dieses Ziel zu erreichen wurde ein Pattern Matching durchgeführt, um ähnliche Mustern zwischen Parametern der \ac{fhir}-Profilen und bestimmten Parametern der Konfigurationsvariablen von \ac{copra} zu erkennen. Für die Mustererkennung wurden \ac{sql}-Abfragen mit \ac{regex} in den WHERE-Bedingungen für jedes Profil entwickelt. Die Entscheidung der Nutzung von \ac{regex} statt des \ac{sql}-Befehls \glqq LIKE\grqq{} basiert auf der Tatsache, dass die \ac{regex} eine breite Palette an Möglichkeiten bietet, um komplexe Muster zu definieren (\ref{sec:regex}).

Dieser Prozess generierte einen Datensatz mit den zugeordneten Konfigurationsvariablen - Biosignaldaten, mit den entsprechenden \ac{fhir}-Profilen. Dieser Datensatz beinhaltet die wichtigsten Attribute beider Systeme und wurde mit Hilfe der Spezialisten der \ac{pdms}-Abteilung validiert.

\subsection{Transformationsregeln} \label{sec:transformrules}

Für die Definition und Programmierung der Transformationsregeln wurden an erster Stelle die Maßeinheiten beider Systeme in dem erzeugten Datensatz nach der Validierung verglichen, um Unregelmäßigkeiten bei diesem Attribut zu detektieren und bestimmte Umwandlungen von Maßeinheiten vorzunehmen. Die Umrechnungen wurden in dem generierten Datensatz des Pattern Matchings eingefügt.

Der resultierende Datensatz wurde in die \ac{copra}-Instanz des Staging Bereichs des \ac{dw} importiert. Mit diesem Datensatz in der \ac{copra}-Instanz, zusammen mit den definierten Spalten mit den Parametern der Biosignaldaten in den Werttabellen, den Spalten mit den Attributen der behandelnden Personen, und Metadaten der Werttabellen, wurden weitere Transformationsregeln für die Überführung der Biosignaldaten aus \ac{copra} in die \ac{fhir}-Profile programmiert.

