\section{Analyse der Maßeinheiten} \label{sec:units}

Bei der Programmierung der Transformationsregeln ist ein wichtiger Aspekt zu berücksichtigen: die Maßeinheiten der Konfigurationsvariablen und der \ac{fhir}-Profile. Diese Einheiten müssen harmonisiert werden. Um dieses Ziel zu erreichen, wurden die Schreibweise und die Dimensionen der physikalischen Größen der Maßeinheiten analysiert. Konfigurationsvariablen bei denen die Dimensionen der Einheiten nicht mit den von den \ac{fhir}-Profilen übereinstimmen oder die Maßeinheiten im \ac{copra}-System nicht dokumentiert sind, wurden nicht berücksichtigt. Werte in den Werttabellen mit Maßeinheiten mit denselben Dimensionen, wie bei den \ac{fhir}-Profilen, aber mit anderen Untereinheiten wurden umgerechnet.

Die Maßeinheiten der \ac{fhir}-Profile wurden bei dem Import in der \ac{db} gleichzeitig analysiert, und die Anmerkungen, wie unterschiedliche Schreibweise derselben Maßeinheit oder Untereinheiten zwischen Profilen, wurden als Issue auf der Webseite von SIMPLIFIER des Moduls \glqq\ac{icu}\grqq{} gemeldet.

Die Konfigurationsvariablen ohne Maßeinheiten wurden den Spezialisten der \ac{pdms}-Abteilung gesendet, denn manche Konfigurationsvariablen beinhalten die Maßeinheiten am Frontend von \ac{copra} und nicht in der \ac{db} des Systems.

Für den Vergleich zwischen den Maßeinheiten der Konfigurationsvariablen von \ac{copra} und der \ac{fhir}-Profile in der \ac{db}, wurde eine \ac{sql}-Abfrage mit integrierten \acp{regex} programmiert. Dieser Schritt ist nicht nur für den Vergleich der Maßeinheiten zwischen beiden Systemen notwendig, sondern auch für die Erkennung von Unregelmäßigkeiten im \ac{copra}-System.

Am Ende dieser Analyse wurde der Datensatz des Pattern Matchings gefiltert, und die zugeordneten Paare von Konfigurationsvariablen und \ac{fhir}-Profile mit Problemen bezüglich der Maßeinheiten wurden herausgenommen. Dieser neue Datensatz wurde in einer neuen Tabelle (\ref{sec:unitscopra}) mit einer neuen Spalte für die Umrechnung der Maßeinheiten für die Harmonisierung in \ac{copra} kopiert. Dieser Datensatz wird wiederum in die \ac{copra}-Instanz des Staging Bereichs des \ac{dw} importiert, um weitere Schritte des Data Mappings durchzuführen.