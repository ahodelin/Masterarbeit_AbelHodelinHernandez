\section{Bereitstellung der \acs{copra}-Daten im \acs{dw}} \label{sec:prepdwtofhir}

Mit der Modellierung der Zuweisung der Felder und der Definition von neuen Transformationsregeln für die Überführung der Biosignaldaten aus \ac{copra} in die \ac{fhir}-Ressourcen des Erweiterungsmoduls \glqq Intensivmedizin\grqq{} wurden drei \ac{sql}-Views programmiert, um die Parameter der Profile, Attribute der Biosignaldaten in den Werttabellen, zusammen mit den dazugehörigen Daten der behandelnden Personen, zu verlinken und zu visualisieren. Die angelegten \ac{sql}-Views sind Folgende:

\begin{itemize}
	\item \texttt{v\_profil\_decimal}: Information der Profile und der Biosignaldaten in der Tabelle \texttt{co6\_data\_decimal\_6\_3}.
	\item \texttt{v\_profil\_string}: Information der Profile und der Biosignaldaten in der Tabelle \texttt{co6\_data\_string}.
    \item \texttt{v\_profil\_pressure}: Information der Profile und der Biosignaldaten in der Tabelle \texttt{co6\_medic\_pressure}.
\end{itemize}

Der Code \ref{list:viewpressure} zeigt den \ac{sql}-Befehl für die Erzeugung der \ac{sql}-View für die Bioparameter in der Tabelle \texttt{co6\_data\_string}. 

\begin{lstlisting}[language=SQL, caption={[SQL-View für Werte in co6\_data\_string] SQL-View für Werte in co6\_data\_string.}, captionpos=b, label=list:viewsdicimalstring]
	create or replace view copra.v_profil_string 
	as
	select 
	's_'||md5(
	(select table_name 
	from information_schema.tables 
	where table_schema = 'copra'
	and table_name = 'co6_data_string') 
	|| cdd.id 
	|| mmc.profile_name) id, -- ID aus der Zusammensetzung von dem Name der Werttabelle, id des Werts und name des Profils
	mmc.meta_profile,
	'final' status,
	mmc.category_coding_system,
	mmc.category_coding_code,
	mmc.code_coding_system_snomed,
	mmc.code_coding_code_snomed,
	mmc.code_coding_system_loinc,
	mmc.code_coding_code_loinc,
	mmc.code_coding_system_ieee,
	mmc.code_coding_code_ieee,
	'p_'||md5(cmdp.id::varchar) subject_reference,
	cdd.val::decimal * mmc.unit_transform "valueQuantity_value",  -- type casting umd Umrechnung
	mmc.valuequantity_system "valueQuantity_system",
	mmc.valuequantity_code "valueQuantity_code",
	cdd.datetimeto "effectiveDataTime"
	from copra.co6_data_string cdd 
	join copra.co6_config_variables ccv 
	on cdd.varid = ccv.id 
	join copra.mapping_mii_co6_2 mmc 
	on mmc.conf_var_id = ccv.id 
	join copra.co6_medic_data_patient cmdp 
	on cmdp.id = cdd.parent_id 
	where not cdd.deleted
	and cdd.validated 
	and cdd.flagcurrent
	and cdd.val ~ '^\d+$|^\d+\.\d+$' -- Kontrolle der nummerischen Struktur der Werte
	;
\end{lstlisting}


Die \ac{sql}-View für die Biosignale in \texttt{co6\_data\_decimal\_6\_3} wird in dieser Arbeit nicht präsentiert, denn diese View ist ähnlich wie die View für die String Werte, aber weniger komplex. Der Grund hierzu ist, dass die detektierten Biosignaldaten in \texttt{co6\_data\_string}, die den \ac{fhir}-Profilen zugeordnet sind, in Wahrheit numerische Einträge sind (\ref{tab:stringvalue}), und somit sollte der Datentyp der Werte der Biosignaldaten umgewandelt werden, jedoch muss zuvor die Struktur des Wertes in der Spalte \texttt{val} der Tabelle \texttt{co6\_data\_string} kontrolliert werden.
\clearpage
\begin{table}[ht]
	\centering 
	\caption[Eintrag in der Tabelle co6\_data\_string]{Beispiel eines Eintrags in der Tabelle co6\_data\_string. Der angezeigte Wert des Biosignals ist eine Zahl.}
	\label{tab:stringvalue}
	\begin{tabular}{|l|l|l|}
		\hline
		\bfseries Profil & \bfseries Konfigurationsvariable & \bfseries Wert \\ \hline
		Linksventrikulaeres Schlagvolumen & SV & 78.1 \\ \hline
	\end{tabular}
\end{table}

In der \ac{sql}-View für die Blutdruckwerte (\ref{list:viewpressure}) werden die Suffixe \glqq systolic\grqq{} für systolisch, \glqq mean\grqq{} für mittel und \glqq diastolic\grqq{} für diastolisch angewendet, um die Zusammensetzung dieser Werte in den zugeordneten \ac{fhir}-Profilen darzustellen (\ref{sec:fhirprofs}). 

\begin{lstlisting}[language=SQL, caption={[SQL-View für Werte in co6\_medic\_pressure] SQL-View für Werte in co6\_medic\_pressure.} , captionpos=b, label=list:viewpressure]
create or replace view copra.v_profil_pressure 
as
	select 
		'pr_'||md5(
			(select table_name 
			from information_schema.tables 
			where table_schema = 'copra'
			and table_name = 'co6_medic_pressure') 
			|| cdd.id 
			|| mmc.profile_name) id,
		mmc.meta_profile,
		'final' status,
		mmc.category_coding_system,
		mmc.category_coding_code,  
		'p_'||md5(cmdp.id::varchar) subject_reference,
		cdd.datetimeto "effectiveDataTime",
		mmc.code_systolic_coding_system_snomed,
		mmc.code_systolic_coding_code_snomed,
		mmc.code_systolic_coding_system_loinc,
		mmc.code_systolic_coding_code_loinc,
		mmc.code_systolic_coding_system_ieee ,
		mmc.code_systolic_coding_code_ieee,
		cdd.systolic "valueQuantity_value_systolic",
		mmc.valuequantity_system "valueQuantity_system_systolic",
		mmc.valuequantity_code "valueQuantity_code_systolic",
		mmc.code_mean_coding_system_snomed,
		mmc.code_mean_coding_code_snomed,
		mmc.code_mean_coding_system_loinc,
		mmc.code_mean_coding_code_loinc,
		mmc.code_mean_coding_system_ieee ,
		mmc.code_mean_coding_code_ieee,
		cdd.mean "valueQuantity_value_mean",
		mmc.valuequantity_system "valueQuantity_system_mean",
		mmc.valuequantity_code "valueQuantity_code_mean",
		mmc.code_diastolic_coding_system_snomed,
		mmc.code_diastolic_coding_code_snomed,
		mmc.code_diastolic_coding_system_loinc,
		mmc.code_diastolic_coding_code_loinc,
		mmc.code_diastolic_coding_system_ieee,
		mmc.code_diastolic_coding_code_ieee,
		cdd.mean "valueQuantity_value_diastolic",
		mmc.valuequantity_system "valueQuantity_system_diastolic",
		mmc.valuequantity_code "valueQuantity_code_diastolic"
	from copra.co6_medic_pressure cdd 
	join copra.co6_config_variables ccv 
		on cdd.varid = ccv.id 
	join copra.mapping_mii_co6_2 mmc 
		on mmc.conf_var_id = ccv.id 
	join copra.co6_medic_data_patient cmdp 
		on cmdp.id = cdd.parent_id 
	where cdd.validated 
	and not cdd.deleted 
	and cdd.flagcurrent
;
\end{lstlisting}

Der Inhalt der Spalten der programmierten \ac{sql}-Views wurde in der vorherige Sektion (\ref{sec:transfer}) geklärt. 

Mit diesen \ac{sql}-Views können \ac{fhir}-Ressourcen wie im Code \ref{list:fhirres} erzeugt werden, denn eine Zeile in der View entspricht einer \ac{fhir}-Ressource im Erweiterungsmodul \glqq Intensivmedizin\grqq{}.

\begin{lstlisting}[caption={[Beispiel einer \acs{fhir}-Ressource aus \acs{copra}] Beispiel einer \acs{fhir}-Ressource einer Blutdruckmessung aus \acs{copra}.},language=JavaScript, label=list:fhirres, captionpos=b]
	{
		"resourceType": "Observation",
		"id": "pr_d43de0fb06ce5b34d7692ad69b59771f",
		"meta": {
			"profile":  [
			"https://medizininformatik-initiative.de/fhir/ext/modul-icu/
			StructureDefinition/pulmonalarterieller-blutdruck"
			]
		},
		"status": "final",
		"category":  [
		{
			"coding":  [
			{
				"system": "http://terminology.hl7.org/
				CodeSystem/observation-category",
				"code": "vital-signs"
			}
			]
		}
		],
		"code": {
			"coding":  [
			{
				"system": "http://snomed.info/sct",
				"code": "75367002",
			}
			]
		},
		"subject": {
			"reference": "p_e139c454239bfde741e893edb46a06cc"
		},
		"effectiveDateTime": "2022-11-25T09:30:11+01:00",
		"component":  [
		{
			"code": {
				"coding":  [
				{
					"system": "http://loinc.org",
					"code": "8406-1"
				},
				{
					"system": "urn:iso:std:iso:11073:10101",
					"code": "150107"
				}
				]
			},
			"valueQuantity": {
				"value": 104,
				"system": "http://unitsofmeasure.org",
				"code": "mm[Hg]"
			}
		},
		{
			"code": {
				"coding":  [
				{
					"system": "http://loinc.org",
					"code": "8432-7"
				},
				{
					"system": "urn:iso:std:iso:11073:10101",
					"code": "150105"
				}
				]
			},
			"valueQuantity": {
				"value": 86,
				"system": "http://unitsofmeasure.org",
				"code": "mm[Hg]"
			}
		},
		{
			"code": {
				"coding":  [
				{
					"system": "http://loinc.org",
					"code": "8377-4"
				},
				{
					"system": "urn:iso:std:iso:11073:10101",
					"code": "150106"
				}
				]
			},
			"valueQuantity": {
				"value": 67,
				"system": "http://unitsofmeasure.org",
				"code": "mm[Hg]"
			}
		}
		]
	}
\end{lstlisting}