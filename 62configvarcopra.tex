\section{Auswahl der Konfigurationsvariablen von \acs{copra}} \label{sec:configvarcopra}

Die Tabelle \texttt{co6\_config\_variables} in Zusammenhang mit den Werttabellen von \ac{copra} wurde untersucht, um die relevanten Konfigurationsvariablen für die Durchführung der Arbeit zu erkennen. Die Tabelle \texttt{co6\_config\_variables} beinhaltet 7409 Konfigurationsvariablen.
\newpage
Die Kriterien für die Wahl der Variablen sind Folgende:

\begin{itemize}
	\item Variablen in Benutzung: Spalte \texttt{deleted} der Tabelle der Konfigurationsvariablen hat den Wert \texttt{null}
	\item Variablen mit validierten Werten in den Werttabellen: Spalte \texttt{validated} hat den Wert \texttt{true}
	\item Variablen mit nicht gelöschten Werten in den Werttabellen: Spalte \texttt{deleted} der Werttabelle hat den Wert \texttt{false}
	\item Variablen mit aktuellen Werten in den Werttabellen: Spalte \texttt{flagCurrent} der Werttabelle hat den Wert \texttt{true}
	\item Variablen mit Patienten- oder Fallbezug: Spalte \texttt{parent} der Tabelle der Konfigurationsvariablen hat die Werte \texttt{1} für Patientenbezug oder \texttt{20} für Fallbezug
	\item Variablen mit 1000 Werten oder mehr in den Werttabellen
\end{itemize}

Um die Konfigurationsvariablen auszuwählen, wurde für jede Werttabelle eine \ac{sql}-Abfrage, wie im Code \ref{list:selectconfigvar}, realisiert.

\begin{lstlisting}[language=SQL, caption={[SQL-Abfrage zur Auswahl der Konfigurationsvariablen] SQL-Abfrage zur Auswahl der Konfigurationsvariablen im COPRA-System. \glqq copra.value\_table cvt\grqq{} stellt die verschiedenen Werttabellen im System dar, also co6\_data\_decimal\_6\_3, co6\_data\_string und co6\_medic\_pressure. \glqq quantity\_in\_value\_table\grqq{} ist ein Alias für die Spalte der Menge an Werten in den Werttabellen.}, captionpos=b, label=list:selectconfigvar]
	select 
	count(cvt.id) quantity_in_value_table, -- Menge an Werten je Konfigurationsvariable in der Werttabelle
	ccv.id, -- ID der Konfiguationsvariable
	ccv."name" -- Name der Konfigurationsvariable
	from copra.co6_config_variables ccv  -- Tabelle der Konfigurationsvariablen 
	join copra.value_table cvt -- eine der Werttabellen
	on cvt.varid = ccv.id -- Verlinkung der Werttabele mit der Tabelle co6_config_variables
	where not cdd.deleted -- nicht geloeschte Daten
	and ccv.parent in (1, 20) -- Patienten- oder Fallbezug
	and not ccv.deleted -- Konfigurationsvariable wird benutzt
	and cvt.flagcurrent -- aktuelle Daten
	and cvt.validated -- validierte Daten
	group by ccv.id, ccv."name" 
	having count(cvt.id) >= 1000 -- 1000 Werte oder mehr in den Werttabellen
	order by quantity_in_value_table -- Menge an Werten absteigend sortiert
	;
\end{lstlisting}

Von den 7409 Konfigurationsvariablen in der \ac{copra}-Instanz wurden 701 als relevante Variablen identifiziert. Unter diesen befinden sich die Biosignalparameter für die Zuordnung mit den \ac{fhir}-Profilen.

Die Anzahl der ausgewählten Konfigurationsvariablen in den Werttabellen für das Pattern Matching der Variablen mit den \ac{fhir}-Profilen für die Erkennung der Biosignaldaten in \ac{copra} ist in der \ref{tab:configvarvaluetables} dargestellt.

\begin{table}[ht]
	\centering  
	\caption[Anzahl der repräsentierten Konfigurationsvariablen in je Werttabelle]{Anzahl der repräsentierten Konfigurationsvariablen in je Werttabelle. Die meiste Konfigurationsvariablen sammeln numerische Werte. * Nicht alle Konfigurationsvariablen der Typ String speichern Biosignaldaten.}
	
	\label{tab:configvarvaluetables}
	\begin{tabular}{|r|l|}
		\hline
		\bfseries Anzahl & \bfseries Werttabelle \\ \hline
		492 & \texttt{co6\_data\_decimal\_6\_3} \\ \hline % 537
		199 * & \texttt{co6\_data\_string} \\ \hline % 203  
		10 & \texttt{co6\_medic\_pressure} \\ \hline % 10
		701 & Gesamt \\ \hline % 750
	\end{tabular}
\end{table}