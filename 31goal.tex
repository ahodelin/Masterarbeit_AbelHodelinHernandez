\section{Ziel der Arbeit} \label{sec:goal}

Mit der vorher präsentierten Problematik ist das Ziel dieser Arbeit die ersten Schritte zu machen, zur Gewährleistung der Interoperabilität der, für längeren Zeitraum in einem \ac{pdms} an der Universitätsmedizin Mainz, gespeicherten Biosignaldaten, denn diese Information liegt weder in einem Standardformat vor noch beinhaltet es standardisierte gesundheitliche Codesysteme. 

Um dieses Ziel zu erreichen wird die Abbildung der Biosignaldaten des \ac{pdms} in den \ac{fhir}-Spezifikationen des Erweiterungsmoduls \glqq Intensivmedizin\grqq{} des Kerndatensatzes der \ac{mii} erstellt. Die Methode wird an erster Stelle für Forschungszwecke eingesetzt und sollte die Grundsteine für die Umsetzung und Etablierung des Prozesses am Ende dieses Projekts legen. 

Zusätzlich werden verschiedene \ac{it}-Werkzeuge und Techniken angewendet, um die Biosignalparameter für die spätere Überführung in das Standardformat \ac{fhir} bereitzustellen; sodass diese Daten in der Zukunft nicht nur lokal, sondern auch national und international datenschutzkonform benutzt werden können. 

Hiermit wird, unter anderem, das Versorgungsniveau von Patienten und Patientinnen ausgewertet und zusammen mit anderen Projekten wird die frühzeitige Erkennung von gesundheitlichen Störungen geschaffen \cite{icukdz}.