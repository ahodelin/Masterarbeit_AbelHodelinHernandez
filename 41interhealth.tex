\section{Interoperabilität} \label{sec:interop}

Der Begriff Interoperabilität kann je nach Anwendungsgebiet unterschiedliche Bedeutungen tragen. Die jedoch meist genutzte Definition ist die Folgende: \glqq Fähigkeit von zwei oder mehreren Systemen oder Komponenten, Informationen auszutauschen und die ausgetauschten Informationen wieder zu nutzen.\grqq{} \cite{interopdef}

Diese Definition beinhaltet zwei Ebenen der Interoperabilität, nämlich den Austausch von Information (technische Interoperabilität) und die Nutzung der ausgetauschten Information (semantische Interoperabilität) \cite{telemedizin}. Die dritte Ebene ist die Prozessinteroperabilität \cite{ehealtOk}. Dies wird erreicht, wenn Menschen gemeinsames Verständnis über ein Netzwerk teilen, Geschäftssys- teme zusammenarbeiten und Arbeitsabläufe koordiniert werden \cite{interop}. Im Gesundheitswesen wird auch von der klinischen Interoperabilität gesprochen \cite{ehealtOk}. Diese Ebene ist die Fähigkeit mehrere klinische Fachkräfte in unterschiedlichen Versorgungsteams einzusetzen, um eine nahtlose Versorgung von Patienten und Patientinnen zu gewährleisten \cite{interop}.

\subsection{Interoperabilität im Gesundheitswesen} \label{subsec:interopgesund}

Seit einigen Jahren eröffnet die steigende Vernetzung zusammen mit der Digitalisierung neue Möglichkeiten in der Patientenbetreuung \cite{telemedizin}. Denn es ermöglicht eine neue Art der Kommunikation zwischen dem Personal des Gesundheitssystems an verschiedenen Standorten als auch die Kommunikation mit Patienten und Patientinnen \cite{ehealtOk}. Solche Kommunikation benötigt eine organisatorische und inhaltliche Harmonisierung der zur Verfügung gestellten Daten in Konkordanz mit den neuesten \acsu{it}-Standards und Schnittstellen, um die Interoperabilität zu gewährleisten \cite{telemedizin}. 

Ein wichtiger Aspekt auf dem Weg der Digitalisierung, und somit der Verbesserung des Gesundheitssystems, sind verbindliche Klassifikationen und Ontologien für eine eindeutige und fachliche Kommunikation \cite{ehealtOk}. Diese interoperablen Mittel sollen beim Austausch von Daten und die Implementierung von \ac{it}-Lösungen umgesetzt werden. Um dieses Ziel zu erreichen, wurde in Deutschland im Jahr 2016 vom \ac{bmbf} die \ac{mii} gegründet \cite{telemedizin}. 

Die Aufgabe der \ac{mii} ist die Verbesserung von Forschungsmöglichkeiten und Patientenversorgung durch \acs{it}-Lösungen \cite{mii}. Damit werden Forschung und Versorgung innerhalb und zwischen den Universitätskliniken, Forschungseinrichtungen, Unternehmen, Krankenkassen und Patientenvertreter vernetzt \cite{telemedizin, mii}. In dieser Vernetzung spielen Schnittstellen medizinischer Inhalte mit \acs{hl7} (\ref{sec:hl7fhir}) eine sehr wichtige Rolle für den Austausch der medizinischen Daten \cite{telemedizin}. 

Die Mitglieder der \ac{mii} haben sich auf einen Kerndatensatz (\ref{sec:miikdz}) geeignet, der auf Basis internationaler Standards, wie \ac{snomedct} (\ref{subsec:snomed}), \ac{loinc} (\ref{subsec:loinc}), etc. entwickelt wurde, um die Interoperabilität zwischen den Standorten zu garantieren \cite{telemedizin, miikdz}. Dieser Datensatz beinhaltet zusätzlich verschiedene Erweiterungsmodule für unterschiedliche Use Cases \cite{mii}. Diese Spezifikation ist verbindlich für die syntaktische und semantische Kodierung des Inhaltes der Module \cite{icukdz}.

Das Kernelement der \ac{mii} sind die Datenintegrationszentren (\acsu{diz}s), dessen Herausforderung die Aufnahme, Zusammenführung und Aufbereitung der Daten aus verschiedenen Systemen, sowie die Sicherstellung von Datenqualität und Datenschutz dieser Daten ist \cite{mii, diz}. Damit können die Daten in Versorgung und Forschung genutzt werden. Um dieses Ziel zu erreichen, werden die Daten standardisiert, wiederverwendbar und austauschbar gemacht \cite{diz}. Jedes Mitglied der \ac{mii} besitzt ein lokales \ac{diz}, in dem die lokalen Daten gespeichert werden. Bei verteilten Machbarkeitsabfragen, wie z. B. die Übermittlung der Menge an Patienten und Patientinnen mit einer bestimmten Diagnose, wird die gespeicherte Information datenschutzkonform bereitgestellt, und in aggregierter Form dann zentral bewertet.