\section{Parameter der \acs{fhir}-Profile des Moduls \acs{icu}} \label{sec:fhirprofs}

Die Information der \ac{fhir}-Profile des Moduls \glqq\ac{icu}\grqq{} des Kerndatensatzes der \ac{mii} wurde aus der Webseite \href{https://www.medizininformatik-initiative.de/Kerndatensatz/Modul_Intensivmedizin/IGMIIKDSModulICU.html}{Medizininformatik Initiative - Modul ICU - ImplementationGuide} herausgenommen. Obwohl dieses Modul zum Abstimmverfahren von 25.02.2022 bis 08.04.2022 freigeben wurde, befindet es sich noch während der Entwicklung dieses Projekts in der ersten stabilen Ballot-Version. Das bedeutet für diese Arbeit, dass Unregelmäßigkeiten in den Profilen detektiert werden können, und als Issue gemeldet werden.

Die Parameter je Profil an der Webseite, nämlich Name, Typ, Attribute der Maßeinheit, Daten der Codesysteme und vorhandene Metrik des Geräts wurden in die Tabelle \texttt{mii\_icu} der \ac{db} \texttt{mii\_copra} eingefügt.

Die meisten \ac{fhir}-Profile unter der Kategorie \glqq Observation\grqq{} besitzen eine ähnliche Struktur, sodass der Import in die Tabelle \texttt{mii\_icu} erleichtert wird. Die Ausnahmen bilden die Profile der Gruppe \glqq Blutdruck Generisch\grqq{} für die Speicherung der Blutdruckmessungen, z. B. \glqq Linksatrialer Druck\grqq{}. Diese Profile sind eine Zusammensetzung von drei separat kodierten \glqq Observations\grqq{}, die dieselben Attribute teilen, nämlich die Spezifikationen für den systolischen, mittleren und diastolischen Blutdruck. In solchen Fällen wurden auch die semantischen Annotationen der systolischen, mittleren und diastolischen Attribute in der Tabelle \texttt{mii\_icu} registriert.

Die Struktur der Tabelle \texttt{mii\_icu} zur Speicherung der Elemente der \ac{fhir}-Profile ist in der \ref{tab:miiicu} dargestellt.

\begin{longtable}{|p{3.5cm}|l|p{6.7cm}|}
	\caption[Struktur der Tabelle mii\_icu]{Struktur der Tabelle \texttt{mii\_icu}. In dem Feld Spalte befinden sich die gegebenen Namen der \ac{fhir}-Elemente der Profile. Der Datentyp ist der Typ mit dem die Daten in der Tabelle \texttt{mii\_icu} gespeichert werden. Die Spalte Information speichert die Beschreibung des Elements.} \label{tab:miiicu}
	\endfirsthead
		\hline
		\rowcolor{lightgray} Spalte & Datentyp & Information \\ \hline
		profile\_id & int & Generierter nummerischer Identifikator des zugeordneten \ac{fhir}-Profils \\ \hline
		profile\_name & varchar & Name des \ac{fhir}-Profils, z. B. Atemzugvolumen-Waehrend-Beatmung \\ \hline
		category\_coding \_system & text & \acsu{url} der Kategorie \glqq Observation\grqq{}. Dies kann ein \ac{snomedct}-\ac{url} sein. \\ \hline
		category\_coding \_code & varchar & \glqq vital-signs\grqq{} oder \ac{snomedct}-ID \\ \hline
		code\_coding \_system\_snomed & text & http://snomed.info/sct \\ \hline 
		code\_coding \_code\_snomed & varchar & \ac{snomedct} des \ac{fhir}-Profils, z. B. 250874002 \\ \hline
		code\_coding \_system\_loinc & text & http://loinc.org \\ \hline
		code\_coding \_code\_loinc & varchar & \ac{loinc}-Code des \ac{fhir}-Profils, z. B. 76222-9 \\ \hline
		code\_coding \_system\_ieee & text & urn:iso:std:iso:11073:10101 \\ \hline
		code\_coding \_code\_ieee & varchar & \ac{iso}/\ac{ieee} 11073-10101\texttrademark{}-Schlüssel des \ac{fhir}-Profils, z. B. 151980 \\ \hline
		valuequantity \_system & text & http://unitsofmeasure.org \\ \hline
		valuequantity\_code & varchar & Maßeinheiten im \ac{fhir}-Profil, z. B. mL \\ \hline
		device\_reference & text & Zuweisung zur Art der Prozedur (gemessen, eingestellt oder erhoben). Diese Prozedur befindet sich in den Profilen der Typ \glqq DeviceMetric\grqq{} \\ \hline
		code\_systolic \_coding\_system \_snomed & text & http://snomed.info/sct \\ \hline
		code\_systolic \_coding\_code \_snomed & varchar & \ac{snomedct}-ID der systolischen Blutdruckmessung, z. B. 271649006 \\ \hline
		code\_systolic \_coding\_system \_loinc & text & http://loinc.org \\ \hline
		code\_systolic \_coding\_code \_loinc & varchar & \ac{loinc}-Code der systolischen Blutdruckmessung, z. B. 8406-1 \\ \hline
		code\_systolic \_coding\_system \_ieee & text & urn:iso:std:iso:11073:10101 \\ \hline
		code\_systolic \_coding\_code \_ieee & varchar & \ac{iso}/\ac{ieee} 11073-10101\texttrademark{}-Schlüssel der systolischen Blutdruckmessung \\ \hline
		code\_mean \_coding\_system \_snomed & text & http://snomed.info/sct \\ \hline
		code\_mean \_coding\_code \_snomed & varchar & \ac{snomedct}-ID der mittleren Blutdruckmessung, z. B. 6797001 \\ \hline
		code\_mean \_coding\_system \_loinc & text & http://loinc.org \\ \hline
		code\_mean \_coding\_code \_loinc & varchar & \ac{loinc}-Code der mittleren Blutdruckmessung, z. B. 8478-0 \\ \hline
		code\_mean \_coding\_system \_ieee & text & urn:iso:std:iso:11073:10101 \\ \hline
		code\_mean \_coding\_code \_ieee & varchar & \ac{iso}/\ac{ieee} 11073-10101\texttrademark{}-Schlüssel der mittleren Blutdruckmessung, z. B. 150019 \\ \hline
		code\_diastolic \_coding\_system \_snomed & text & http://snomed.info/sct \\ \hline
		code\_diastolic \_coding\_code \_snomed & varchar & \ac{snomedct}-ID der diastolischen Blutdruckmessung, z. B. 271650006 \\ \hline
		code\_diastolic \_coding\_system \_loinc & text & http://loinc.org \\ \hline
		code\_diastolic \_coding\_code \_loinc & varchar & \ac{loinc}-Code der diastolischen Blutdruckmessung, z. B. 8462-4 \\ \hline
		code\_diastolic \_coding\_system \_ieee & text & urn:iso:std:iso:11073:10101 \\ \hline
		code\_diastolic \_coding\_code \_ieee & varchar & \ac{iso}/\ac{ieee} 11073-10101\texttrademark{}-Schlüssel der diastolischen Blutdruckmessung, z. B. 150018 \\ \hline
		meta\_profile & text &  \ac{url} um das Profil zu identifizieren \\ \hline
\end{longtable}
