\begin{abstract}	
	Der Prozess der Digitalisierung im Gesundheitswesen ist ein wichtiger Bestandteil eines sozialen Wandlungsprozesses. Eine der zentralen Herausforderungen in diesem Prozess ist die mangelnde Interoperabilität vieler Systeme, zusammen mit dem gewaltigen generierten Datenvolumen während der Routineversorgung, denn die erfassten Parameter werden nicht immer digital erfasst und werden auch in diversen getrennten Systemen gesteuert.
	
	Ziel dieser Arbeit ist die Bereitstellung der in einem \ac{pdms} gespeicherten Biosignaldaten aus der Routineversorgung eines deutschen Universitätsklinikums für die Überführung dieser Bioparameter in das etablierte Standardformat \ac{fhir}, sodass die Interoperabilität der Biosignaldaten gewährleistet wird, und diese Daten in der Zukunft integriert und wieder anwendbar werden können. Dazu werden verschiedene informatische Werkzeuge angewendet, um die gespeicherten Bioparameter mit den Parametern der definierten \ac{fhir}-Spezifikationen des Moduls \glqq Intensivmedizin\grqq{} der \ac{mii} zusammenzuführen. Am Ende dieser Zusammenführung entsteht ein Grundgerüst für die zukünftige Überführung der Biosignaldaten in \ac{fhir}.	
\end{abstract}