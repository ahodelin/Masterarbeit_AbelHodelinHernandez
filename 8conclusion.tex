\chapter{Fazit und Ausblick} \label{ch:conclussion}

Das Ziel dieser Arbeit wurde erreicht, nämlich die Zuordnung der gespeicherten Biosignaldaten aus dem \ac{pdms} mit den \ac{fhir}-Profilen des Erweiterungsmoduls \glqq Intensivmedizin\grqq{} und die Bereitstellung dieser Daten für die Überführung in \ac{fhir}-Ressourcen. Somit entstand ein Grundgerüst, was für die Überführung der Biosignaldaten in \ac{fhir}-Ressource benutzt werden könnte. Diese Schritte sind entscheidend für die Gewährleistung der Interoperabilität der Biosignaldaten in \ac{copra}, und sind somit notwendige Komponente der Datenmigration und Datenintegration.

Die Relevanz dieses Projekts liegt daran, dass die Biosignalparameter aus dem \ac{copra}-System der Universitätsmedizin Mainz mit Hilfe von der praktischen Umsetzung des Data Mappings in ein Standardformat überführt werden können. Das hat zufolge, dass diese Daten für Forschungszwecke weiterbenutzt werden können. Damit können die standardisierten Daten unter anderem für Datenauswertungen und Reports für Anwendungszwecke, Statistik, Anwendungsszenarien, oder die Entwicklung von Systemen für die Unterstützung von klinischen Entscheidungen angewendet werden.

Ein weiterer wichtiger Aspekt dieses Projekts ist, dass mit der Überführung der Biosignaldaten in ein Standardformat mit standardisierten Codesystemen, die Daten am Ende auch die \ac{fair}-Prinzipien erhalten werden.
