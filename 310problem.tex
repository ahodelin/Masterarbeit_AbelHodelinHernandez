\section{Problemstellung} \label{sec:problem}

In den Krankenhäusern wird eine Unmenge an Daten aus der Routineversorgung gesammelt. Unter diesen Daten befinden sich die Biosignale. Diese sind die Ergebnisse von Messungen oder Beobachtungen. Diese Biosignaldaten werden entweder manuell (Körpergröße, Kopfumfang) oder automatisch durch am Netz verbundene Geräte (Beatmungsdruck, Blutflussindex) erfasst. Die erfassten Parameter werden im Idealfall digital in komplexen Systemen gespeichert. Diese Informationen werden in den meisten Fällen über längere Zeiträume aufgehoben, weil sie entscheidend für die therapeutischen Behandlungen sind. 

Mit der Zeit werden in einem Krankenhaus neue Geräte angeschafft, das Krankenhauspersonal wird erneuert und neue Techniken werden angewendet, sodass bei jeder Erneuerung auch Änderungen in dem System der Speicherung der Daten vorgenommen werden. Außerdem besitzt jedes Krankenhaus seine eigene Systemlandschaft zur Steuerung der Informationen und viele dieser Systeme benutzen keine Standardformate oder Codesysteme bei der Speicherung und Übermittlung der Daten, und somit sind die Daten in den Gesundheitseinrichtungen nicht interoperabel. 

Dadurch, dass die Erfassung der Biosignaldaten in den meisten Fällen nicht den etablierten Standards entspricht, könnten solche Daten mit der Zeit unbrauchbar werden oder im schlimmsten Fall verloren gehen. Diese Situation erschwert die Nutzbarkeit der Informationen für die Forschung und Krankenversorgung - nicht nur an einem Standort, sondern auch deutschlandweit.