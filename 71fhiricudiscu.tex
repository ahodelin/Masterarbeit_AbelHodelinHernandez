\section{\acs{fhir}-Profile} \label{sec:fhiricudisc}

Obwohl die \ac{fhir}-Profile des Moduls \glqq\ac{icu}\grqq{} für die Interoperabilität dienen sollten, beinhalten sie während der Durchführung dieses Projekts einige Irregularitäten, da das Modul in dieser Zeit in Abstimmung sich befindet, und weitere Untersuchungen und Tests von den \ac{mii}-Mitgliedern und anderen Benutzern erfolgen. Aus diesem Grund beinhalten zwei \ac{fhir}-Profile im Erweiterungsmodul denselben Namen, aber spezifizieren unterschiedlichen Beobachtungen.

Die meisten definierten \ac{fhir}-Profile gehören zu der Kategorie \glqq Observation\grqq{} (\ref{tab:proficu}). Das liegt daran, dass das Ziel des Moduls die Spezifikationen der akutmedizinischen Daten ist, und gerade im Bereich der Intensivmedizin verlaufen die meisten Beobachtungen und Messungen (\glqq Observation\grqq{}).

Ein wichtiger Parameter für die Einhaltung der Interoperabilität im Gesundheitswesen sind die Codesysteme (\ref{tab:profilcodes}). Alle nicht generischen Profile beinhalten zumindest ein standardisiertes Codesystem. Die Profile der Kategorie \glqq Observation\grqq{} ohne Codesysteme gehören zu den generischen Profilen (\ref{fig:icutreegenerics}) und dienen der Modellierung der Daten (\ref{subsec:icumodul}). Aus diesem Grund beinhalten solche Profile keine Codesysteme und auch keine Maßeinheiten (\ref{tab:profilnocode}). Die Profile der Kategorie \glqq Procedure\grqq{} beinhalten auch keine Codesysteme, denn diese Profile speichern die Information der angewandten Geräte für die Durchführung eines Verfahrens und diese Information befindet sich im Basismodul \glqq Prozedur\grqq{} der Kerndatensatz der \ac{mii}.

\begin{figure}[ht]
	\centering
	\includegraphics[height=8cm]{figures/icu_modul_tree_generics}
	\caption[Generische \glqq Observation\grqq{}-Profile]{Baumstruktur des Erweiterungsmoduls \glqq Intensivmedizin\grqq{}. Die generischen Profile der Kategorie \glqq Observation\grqq{} sind rot markiert.}
	\label{fig:icutreegenerics}
\end{figure}

Das Modul \glqq\ac{icu}\grqq{} mit 80 \ac{fhir}-Profilen ist das umfangreichste Modul der \ac{mii}. Von diesen Profilen könnte in diesem Projekt 39 einem Biosignalparameter zugeordnet werden (\ref{fig:profile}). Die Gründe dieses Ergebnisses sind nicht nur von dem Umfang des Moduls abhängig, sondern von den Bioparameter im \ac{copra}-System (\ref{sec:configvarcopradiscu}).

\clearpage

\begin{figure}[ht]
	\centering
	\includegraphics[height=7cm]{figures/profile}
	\caption[Diagramm der \acs{fhir}-Profile im Projekt]{Diagramm der \acs{fhir}-Profile im Projekt. Fast die Hälfte der \ac{fhir}-Profile konnten zugeordnet werden.}
	\label{fig:profile}
\end{figure}
