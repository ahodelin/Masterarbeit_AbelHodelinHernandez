\section[Data Mapping der Biosignaldaten aus \acs{copra} mit den \acs{fhir}-Profilen]{Data Mapping der Biosignaldaten aus \\ \acs{copra} mit den \acs{fhir}-Profilen} \label{sec:datamappingicucopra}

Nach der Analyse der \ac{fhir}-Profile des Erweiterungsmoduls \glqq Intensivmedizin\grqq{} und des \ac{copra}-Datenmodells wurden die Information beider Systeme für die Datendefinition analysiert und ein Pattern Matching-Prozess für die Zuordnung der Konfigurationsvariablen aus \ac{copra} mit den \ac{fhir}-Profilen durchgeführt. Während dieses Verfahrens entstand ein Datensatz. Dieser wurde validiert und für die Zuweisung der Quell- und Zielfelder, und die Programmierung der Transformationsregeln benutzt.

\subsection{Datendefinition} \label{subsec:datadef}

Die Tabellen mit den Attributen der \ac{fhir}-Profile (\ref{sec:fhirprofs}) und die Tabellen von \ac{copra} (\ref{sec:configvarcopra}) wurden untersucht, um zu erkennen, welche \ac{copra}-Tabellen und Spalten mit den Elementen der \ac{fhir}-Profile zu vernetzen sind.

Die Tabelle der \ac{fhir}-Profile (\ref{tab:miiicu}) beinhaltet die Information der Elemente der Profile des Moduls \glqq\ac{icu}\grqq{} des Kerndatensatzes der \ac{mii} und somit die Datenstruktur und Parameter des Zielsystems.

Das Quellsystem dieser Arbeit sind die Tabellen der \ac{copra}-Instanz des Staging Bereichs des \ac{dw} des \ac{diz} an der Universitätsmedizin Mainz. In der Tabelle der Konfigurationsvariablen (\ref{tab:co6confvar}) sind die am Standort angegebenen \ac{copra}-Namen und Beschreibungen der Messungen oder angewandten Techniken zusammen mit deren Maßeinheiten zu finden. Die Tabelle der Typen der Konfigurationsvariablen (\ref{tab:co6confvartype}) beinhaltet die Datentypen der Konfigurationsvariablen und die Namen der Werttabellen, in den die Werte der Biosignaldaten gespeichert sind. Eine andere wichtige Tabelle in \ac{copra} ist die mit den Basisdaten der behandelnden Personen (\ref{tab:patient}). Die Werte und Zeitangaben der Messungen befinden sich in den Werttabellen (\ref{tab:valuetab}, \ref{tab:valuepress}).

Für das Weiterlaufen des Projekts wurden nur die Konfigurationsvariablen ausgewählt, die einen Patientenbezug oder Fallbezug besitzen, mit mindestens 1000 validierten, aktuellen und nicht gelöschten Datensätzen im \ac{copra}-System, denn unter dieser Gruppe befinden sich die Konfigurationsvariablen, die den Biosignaldaten entsprechen.

\subsection{Zuweisung der Quell- und Zielfelder} \label{sec:patternmatchingicucopra}

Nach der Definition der Daten wurden die Quell- und Zielfelder zugewiesen. Um diese Zuweisung zu erzielen wurde ein Pattern Matching durchgeführt, um ähnliche Muster zwischen den Parametern der \ac{fhir}-Profile und bestimmten Attributen der Konfigurationsvariablen zu erkennen. 

Zu Beginn wurden die Namen der \ac{fhir}-Profile zusammen mit den Namen und Beschreibungen der Konfigurationsvariablen mit einer unscharfen Suche mit Hilfe von der Bibliothek \glqq thefuzz\grqq{} in Python analysiert. Diese unscharfe Suche liefert einen Datensatz mit Paaren von Profilen und Konfigurationsvariablen. Anschließend wurden \ac{sql}-Abfragen mit \acp{regex} für jedes Profil entwickelt, um falsch zugeordnete Konfigurationsvariablen herauszunehmen und nicht erkannte Konfigurationsvariablen wahrzunehmen. Die Entscheidung der Nutzung von \acp{regex} statt des \ac{sql}-Befehls \glqq LIKE\grqq{} basiert auf der Tatsache, dass die \acp{regex} eine breite Palette an Möglichkeiten bietet, um komplexe Muster zu definieren (\ref{sec:regex}).

Dieser Prozess generierte einen Datensatz mit den zugeordneten Paaren: Konfigurationsvariable - \ac{fhir}-Profil. Dieser Datensatz beinhaltet die wichtigsten Attribute beider Systeme und wurde mit Hilfe der Spezialisten der \ac{pdms}-Abteilung validiert.

Mit dem validierten Datensatz wurden die Attribute der \ac{fhir}-Profile mit den Felder der Tabellen von \ac{copra} zugewiesen, und die Transformationsregel wurden definiert und programmiert.

\subsection{Transformationsregeln} \label{sec:transformrules}

Für die Definition und Programmierung der Transformationsregeln wurden an erster Stelle die Maßeinheiten beider Systeme im erzeugten Datensatz nach der Validierung verglichen, um Unregelmäßigkeiten bei diesem Attribut zu detektieren und bestimmte Umwandlungen für die Harmonisierung der Maßeinheiten vorzunehmen. Die Umrechnungen wurden im generierten Datensatz des Pattern Matchings eingefügt.

Der resultierende Datensatz wurde in die \ac{copra}-Instanz des Staging Bereichs des \ac{dw} importiert. Mit diesem Datensatz in der \ac{copra}-Instanz, zusammen mit den definierten Spalten mit den Parametern der Biosignaldaten in den Werttabellen, den Spalten mit den Attributen der behandelnden Personen, und Metadaten, wurden weitere Transformationsregeln für die Überführung der Biosignaldaten aus \ac{copra} in \ac{fhir} programmiert.