\chapter{Einleitung} \label{ch:introduction}

Ein wichtiger Bestandteil eines sozialen Wandlungsprozesses ist der Prozess der Digitalisierung im Gesundheitswesen, denn die generierte Datenmenge ist heutzutage nicht im Papierformat zu bewältigen. Dieser Prozess bringt Herausforderungen mit sich, die zu bewältigen sind, sodass die Nutzung moderner \ac{it}-Technologien und Standards im Gesundheitswesen in Bezug auf eine Verbesserung der Versorgung und Forschung im Gesundheitssystem ermöglicht wird. Hinzukommend übt die Digitalisierung auch einen Einfluss auf die Entwicklung der Interaktion zwischen unterschiedlichen an der Gesundheitsversorgung beteiligten Instanzen aus. 

Eine der zentralen Herausforderungen im Prozess der Digitalisierung im Gesundheitswesen ist, zusammen mit dem immens generierten Datenvolumen, die mangelnde Interoperabilität vieler Systeme. Denn viele Unternehmen haben eigene Lösungen für einzelne Komponenten hergestellt, sodass die Interaktion von Systemen an einem Standort oder die Kommunikation zwischen verschiedenen Standorten in vielen Fällen impraktikabel ist. Eine weitere Problematik der mangelnden Interoperabilität ist die dazu mangelnde Nutzbarkeit der Daten für die Versorgung sowie für die Forschung, die auch die Krankenversorgung in näherer Zukunft fördern könnte. Aus diesem Grund soll innerhalb dieser Masterarbeit der Weg für die Erschließung von Biosignaldaten aus der Routineversorgung an der Universitätsmedizin Mainz primär für die wissenschaftliche Nutzung aufgezeigt werden.