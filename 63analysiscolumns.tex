\section{Analyse der Tabellen für die Datendefinition} \label{sec:analysiscolums}

Mit der Information der relevanten Parameter von \ac{copra} zusammen mit den Parametern der \ac{fhir}-Profile, wurden die geeigneten Spalten der Tabellen von \ac{copra} und der Tabelle \texttt{mii\_icu} in der \ac{db} \texttt{mii\_copra} ausgewählt, die die Parameter für die Überführung der Biosignaldaten in \ac{fhir} beinhalten.

Die Spalten \texttt{profile\_name} und \texttt{loinc} in der Tabelle \texttt{mii\_icu} (\ref{tab:miiicu}), und die Spalten \texttt{name} und \texttt{description} in der Tabelle \texttt{co6\_config \_variables} (\ref{tab:co6confvar}) wurden als geeignete Kandidaten ausgewählt, um die Konfigurationsvariablen aus \ac{copra} und den \ac{fhir}-Profilen miteinander zu verlinken, denn diese Spalten beinhalten die Hauptinformation für den späteren Zuordnungsprozess.
 
 Einerseits beinhaltet die Spalte \texttt{profile\_name} die Namen der \ac{fhir}-Profile, die wiederum auch die Namen von Verfahren oder Messungen sind, andererseits registriert die Spalte \texttt{loinc} die \ac{loinc}-Codes. Der Vorteil dieses Codesystem anzuwenden, liegt daran, dass \ac{loinc} das meist verwendete Codesystem in den \ac{fhir}-Profilen (\ref{tab:profilcodes}) ist. 
 
 Nach der Analyse der gespeicherten \ac{loinc}-Codes wurden Abkürzungen gefunden, die für den späteren Pattern Matching-Prozess von Bedeutung sind, denn sie dienen als zusätzliche Namen \glqq Short Name\grqq{} in der Beschreibung der \ac{loinc}-Codes. Manche dieser \glqq Short Names\grqq{} werden auch in der alltäglichen Kommunikation unter dem Gesundheitspersonal benutzt, wie die Abkürzung FIO2 für die Sauerstofffraktion (\ref{tab:shortname}). Solche \glqq Short Names\grqq{} mit einigen Variationen wurden auch in den Namen und Beschreibungen von den Konfigurationsvariablen in \ac{copra} durch das Pattern Matching erkannt.

\begin{table}[ht]
	\centering 
	\caption[Beispiel von \acs{loinc}-\glqq Short Name\grqq{}]{Beispiel von \acs{loinc}-\glqq Short Name\grqq{}. Das \ac{fhir}-Profil \glqq Sauerstofffraktion\grqq{} hat den \ac{loinc}-Code, \href{https://loinc.org/71835-3/}{71835-3}. Dieser wiederum besitzt den \glqq Short Name\grqq{} FIO2.}
	\label{tab:shortname}
	\begin{tabular}{|p{4cm}|l|l|}
		\hline
		\bfseries Profilname & \bfseries \ac{loinc}-URL & \bfseries Short Name \\ \hline
		Sauerstofffraktion & \url{https://loinc.org/71835-3/} & 
		FIO2 \\ \hline 
	\end{tabular}
\end{table}

In der Tabelle \texttt{co6\_config\_variables} sind die an dem Standort angegebenen \ac{copra}-Namen für Messungen oder Techniken. Diese sind in der Spalte \texttt{name} registriert. 

Die \ac{copra}-Namen können Fachtermini oder eine Zusammensetzung von Namensabkürzungen des angewandten Verfahrens und Gerätes sein. Manche der Abkürzungen des Verfahrens sind Varianten des vorher erläuterten \ac{loinc}-\glqq Short Name\grqq{}. Die Spalte \texttt{description} in \texttt{co6\_config\_variables} beinhaltet eine kurze Beschreibung der Verfahren und in manchen Fällen befindet sich in dieser Beschreibung der ganze Name eines Verfahrens. Diese Details führten dazu, dass die Spalte \texttt{description} für die Zuordnung der Konfigurationsvariablen mit den \ac{fhir}-Profilen betrachtet werden sollte.

Die Maßeinheiten der Konfigurationsvariablen und somit der Biosignaldaten befinden sich in der Spalte \texttt{unit} der Tabelle \texttt{co6\_config\_variables}. Andererseits befinden sich die Maßeinheiten der Profile in der Spalte \texttt{valuequan tity\_code} der Tabelle \texttt{mii\_icu}. Die Maßeinheiten sind entscheidend, da diese in beiden Systemen für die Harmonisierung vergleichbar sein müssen. Das bedeutet, dass die Maßeinheiten beider Systeme dieselben Dimensionen der physikalischen Größen besitzen müssen, sodass eine Umrechnung zwischen den Einheiten beider Systeme möglich ist.

Dadurch, dass ein Identifikator der Patienten in den \ac{fhir}-Profilen als Referenz zu der \ac{fhir}-Ressource \glqq Patient\grqq{} im Kerndatensatz der \ac{mii} benutzt wird, um die Biosignaldaten zu den korrespondierenden, behandelnden Personen zu verlinken, wird dieser Parameter vom \ac{copra}-System benötigt. Diese Information befindet sich in der Tabelle \texttt{co6\_medic\_patient} (\ref{tab:patient}) in der Spalte \texttt{patid}. Der Inhalt dieser Spalte ist ein Pseudonym der Patientennummer im \ac{kis}. Auch in der Tabelle \texttt{co6\_medic\_patient} soll die Spalte \texttt{id} berücksichtigt werden, denn diese ist der primäre Schlüssel dieser Tabelle und ist somit Fremdschlüssel in den Werttabellen. Damit werden die Daten der Patienten mit ihren Biosignaldaten im \ac{copra}-System verknüpft. Die Spalte \texttt{deleted} in der Tabelle \texttt{co6\_medic\_patient} wird benötigt, um über die Löschung einer Person aus dem System in Kenntnis gesetzt zu werden. Die Abbildung von \texttt{patid} in den \ac{fhir}-Profilen ist das Attribut \texttt{reference} des Elements \texttt{subject}.

Die Werte und Zeitangaben der Biosignale befinden sich in den Werttabellen. Diese Werte sind in der Spalte \texttt{val} der Tabellen \texttt{co6\_data\_decimal\_6\_3} und \texttt{co6\_data\_string} (\ref{tab:valuetab}), und in den Spalten \texttt{systolic}, \texttt{mean} und \texttt{diastolic} der Tabelle \texttt{co6\_medic\_pressure} (\ref{tab:valuepress}) zu finden. Die Zeiten der Messungen sind in der Spalte \texttt{datetimeto} aller Werttabellen zu finden. Die Abbildung der Werte und Zeitangaben der Biosignale in den \ac{fhir}-Profilen sind das Attribut \texttt{value} des Elements \texttt{valueQuantity} beziehungsweise das Element \texttt{effectiveDateTime}.

 Die Werttabellen beinhalten den Hauptschlüssel der Tabelle \texttt{co6\_medic \_patient} in der Spalte \texttt{parent\_id} als Fremdschlüssel, wenn der Wert dieser Spalte gleich 1 ist. Damit wird die Information der Biosignaldaten den behandelnden Personen zugewiesen. Die Spalte \texttt{var\_id} ist der Schlüssel, welcher die Werte den korrespondierenden Konfigurationsvariablen in der Tabelle \texttt{co6\_config\_variables} zuweist. 
 Der Hauptschlüssel einer Werttabelle in der Spalte \texttt{id} in Kombination mit dem Namen der Werttabelle wird benutzt, um einen Ersatzschlüssel für die Überführung der Daten von \ac{copra} in \ac{fhir} zu erstellen.

Nach der Datendefinition wird die Zuordnung der Konfigurationsvariablen mit den \ac{fhir}-Profilen durch das Pattern Matching durchgeführt. Das Ergebnis dieses Prozesses wird in der Tabelle \texttt{mapping\_mii\_co6} (\ref{tab:mapping}) gespeichert.

\begin{longtable}{|p{3.5cm}|l|p{6.5cm}|}
	\caption[Tabelle für die Speicherung der Zuordnung der Konfigurationsvariablen mit den \acs{fhir}-Profilen]{Tabelle für die Speicherung der Zuordnung der Konfigurationsvariablen mit den \acs{fhir}-Profilen nach dem Pattern Matching. Die Spalten mit \texttt{coding\_system} in den Namen speichern die \ac{uri} (Anhang \ref{sec:uri}) der Terminologie-Systeme.}
	\label{tab:mapping}
	\endfirsthead
	\hline			
		\bfseries Spalte & \bfseries Datentyp & \bfseries Information \\ \hline		
		profile\_id & int & Generierter numerischer Identifikator des zugeordneten \ac{fhir}-Profils \\ \hline
		profile\_name & varchar & Name des \ac{fhir}-Profils, z. B. Atemzugvolumen-Waehrend-Beatmung \\ \hline
		category\_coding \_system & text & \acsu{url} der Kategorie. Dies kann ein \ac{snomedct}-\ac{url} bei einigen \glqq Observations\grqq{} sein. \\ \hline
		category\_coding \_code & varchar & \glqq vital-signs\grqq{} oder \ac{snomedct}-ID \\ \hline
		code\_coding \_system\_snomed & text & http://snomed.info/sct \\ \hline 
		code\_coding \_code\_snomed & varchar & \ac{snomedct} des \ac{fhir}-Profils, z. B. 250874002 \\ \hline
		code\_coding \_system\_loinc & text & http://loinc.org \\ \hline
		code\_coding \_code\_loinc & varchar & \ac{loinc}-Code des \ac{fhir}-Profils, z. B. 76222-9 \\ \hline
		code\_coding \_system\_ieee & text & urn:iso:std:iso:11073:10101 \\ \hline
		code\_coding \_code\_ieee & varchar & \ac{iso}/\ac{ieee} 11073-10101\texttrademark{}-Schlüssel des \ac{fhir}-Profils, z. B. 151980 \\ \hline
		valuequantity \_system & text & http://unitsofmeasure.org \\ \hline
		valuequantity\_code & varchar & Maßeinheiten im \ac{fhir}-Profil, z. B. mL \\ \hline
		conf\_var\_unit & varchar & Maßeinheiten der Konfigurationsvariable in \ac{copra}, z. B. ml \\ \hline
		device\_reference & text & Zuweisung zur Art der Prozedur (gemessen, eingestellt oder erhoben). Diese Prozedur befindet sich in den Profilen des Typs \glqq DeviceMetric\grqq{} \\ \hline
		meta\_profile & text &  \ac{url} um das Profil zu identifizieren \\ \hline
		conf\_var\_id & bigint & Primärer Schlüssel der Tabelle \texttt{co6\_config\_variables}, z. B. 104726 \\ \hline
		conf\_var\_parent\_id & bigint & Schlüssel des Bezuges der Variable, z. B. 1 für Patientenbezug \\ \hline
		conf\_var\_parent \_name & varchar & Name des Bezuges, z. B. Patient \\ \hline
		conf\_var\_name & varchar & Name der Konfigurationsvariable, z. B. Beatmung\_MS\_Pallas\_Vt \\ \hline
		conf\_var\_description & varchar & Beschreibung der Konfiguratiovsvariable, z. B. gemessenes Tidalvolumen \\ \hline
		conf\_var\_types\_id & int & Schlüssel des Datentyps in der Tabelle \texttt{co6\_config\_variable\_types}, z. B. 6 für nummerische Werte \\ \hline
		conf\_var\_types\_name & varchar & Name des Datentyps in der Tabelle \texttt{co6\_config\_variable\_types}, z. B. Decimal\_6\_3 \\ \hline
		code\_systolic \_coding\_system \_snomed & text & http://snomed.info/sct \\ \hline
		code\_systolic \_coding\_code \_snomed & varchar & \ac{snomedct}-ID der systolischen Blutdruckmessung, z. B. 271649006 \\ \hline
		code\_systolic \_coding\_system \_loinc & text & http://loinc.org \\ \hline
		code\_systolic \_coding\_code \_loinc & varchar & \ac{loinc}-Code der systolischen Blutdruckmessung, z. B. 8406-1 \\ \hline
		code\_systolic \_coding\_system \_ieee & text & urn:iso:std:iso:11073:10101 \\ \hline
		code\_systolic \_coding\_code \_ieee & varchar & \ac{iso}/\ac{ieee} 11073-10101\texttrademark{}-Schlüssel der systolischen Blutdruckmessung, z. B. 150107 \\ \hline
		code\_mean \_coding\_system \_snomed & text & http://snomed.info/sct \\ \hline
		code\_mean \_coding\_code \_snomed & varchar & \ac{snomedct}-ID der mittleren Blutdruckmessung, z. B. 6797001 \\ \hline
		code\_mean \_coding\_system \_loinc & text & http://loinc.org \\ \hline
		code\_mean \_coding\_code \_loinc & varchar & \ac{loinc}-Code der mittleren Blutdruckmessung, z. B. 8478-0 \\ \hline
		code\_mean \_coding\_system \_ieee & text & urn:iso:std:iso:11073:10101 \\ \hline
		code\_mean \_coding\_code \_ieee & varchar & \ac{iso}/\ac{ieee} 11073-10101\texttrademark{}-Schlüssel der mittleren Blutdruckmessung, z. B. 150019 \\ \hline
		code\_diastolic \_coding\_system \_snomed & text & http://snomed.info/sct \\ \hline
		code\_diastolic \_coding\_code \_snomed & varchar & \ac{snomedct}-ID der diastolischen Blutdruckmessung, z. B. 271650006 \\ \hline
		code\_diastolic \_coding\_system \_loinc & text & http://loinc.org \\ \hline
		code\_diastolic \_coding\_code \_loinc & varchar & \ac{loinc}-Code der diastolischen Blutdruckmessung, z. B. 8462-4 \\ \hline
		code\_diastolic \_coding\_system \_ieee & text & urn:iso:std:iso:11073:10101 \\ \hline
		code\_diastolic \_coding\_code \_ieee & varchar & \ac{iso}/\ac{ieee} 11073-10101\texttrademark{}-Schlüssel der diastolischen Blutdruckmessung, z. B. 150018 \\ \hline
		matching\_valide & bool & Ergebnis der Validierung der Zuordnung, \texttt{true} oder \texttt{false} \\ \hline
\end{longtable}