\section{PostgreSQL-\acs{db}} \label{sec:database}

Für die Entwicklung dieses Projekts wurde die \ac{db} \texttt{mii\_copra} in PostgreSQL erstellt, um Attribute der \ac{fhir}-Profile und Parameter der im \ac{copra} gespeicherten Konfigurationsvariablen zu speichern. In dieser \ac{db} fanden auch die Bearbeitung und Analyse der Daten statt.

Die \ac{db} \texttt{mii\_copra} wurde in einer Testumgebung mit PostgreSQL 14.2 unter Linux Ubuntu 22.04.1 \ac{lts} implementiert. Für die Entwicklung der \ac{db} und die weiteren \ac{sql}-Prozesse wurde die Community Version des freien Open Source Werkzeugs \href{https://dbeaver.io/}{DBeaver} benutzt.

In der \ac{db} \texttt{mii\_copra} wurde eine Tabelle angelegt für die Speicherung bestimmter herausgenommener Parameter der \ac{fhir}-Profile der \ac{mii} - Modul \glqq\ac{icu}\grqq{}. Zwei weitere Tabellen sollen die relevanten Attribute der Metadaten von \ac{copra} speichern. Diese Tabellen mit Metadaten 
sammeln die Informationen der Konfigurationsvariablen und Konfigurationsvariablen-Typen und beinhalten unter anderem die Namen und Datentypen der an der Intensivstation und Notaufnahme der Universitätsmedizin Mainz gemessenen Biosignaldaten. In der \ac{db} wurden auch die \ac{sql}-Abfragen für die Zusammenführung der \ac{fhir}-Profile mit den Biosignaldaten geschrieben. Schließlich wurden weitere Tabellen angelegt, um die resultierenden Zwischenergebnisse und Datensätze zu speichern, analysieren und visualisieren (\ref{tab:miiicu}, \ref{tab:mapping}).